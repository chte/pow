The architecture of the testing framework may be described in terms of data flow (protocol) and control flow (software). On the following pages both data flow and control flow of the application will be thoroughly explained.

\subsection{Protocol}

\subsubsection{Protocol notation}\label{text:protonot}
To formalize the Proof of Work protocol a few notations will be introduced. Consider the following notations:
\\
\\
\indent \begin{minipage}{0.9\linewidth}
\thickmuskip=0mu
${\textbf M_i} \ = $ to be $i^{th}$ execution of the protocol $M$ by either a legitimate user or an adversary. 

${\textbf h} \ = $ a puzzle generator function.

${\textbf t} \ = $ timestamp at time of puzzle generation

${\textbf P[j]} \ = $ the $j^{th}$ sub-puzzle in $P$.

${\textbf m} \ = $ the number of sub-puzzles, this equals to the size of the set $P$. 

${\textbf d} \ = $ an integer indicating the difficulty of the problem set $P$. 

${\textbf S[j]} \ = $ the $j^{th}$ solution in $S$ and a solution to $j^{th}$ sub-puzzle in $P$. 

${\textbf z} \ = $ a function that return the number of most significant bytes that is zero. 
%\vspace*{-10pt}
\end{minipage}


\subsubsection{Protocol description}\label{text:protodesc}
Now let us describe the details of our proposed reputation based puzzle protocol between client and server. Prior to initiating protocol \textbf{M} the client $C_i$ starts by requesting for a service $R_i$ to the server. The server responds with a packaged set of sub-puzzles $P_i$ of size $m$ and a difficutly $d$. Each sub-puzzle $P[1] \dots P[m]$ is a seed derived by $g(t,x)$ where $x$ is value generated uniformly-at-random.

The client $C_i$ must solve each sub-puzzle $P_i[j]$ of $P_i$ by finding a value $S_i[j]$ so that the computed hash $H(S_i[j] \ || \ P_i[j] )$ has at least $d$ leading zeroes. If such a hash is found then the solution $S[j]$ is a solution to the sub-puzzle $P[j]$.

The server then verifies each solution of $S_i$ by computing $H(S_i[j] \ || \ P_i[j])$ with a stored copy of $P_i[j]$ and confirms that the solution has $d$ leading zeroes. If all solutions are correctly computed the client $C_i$'s request $R_i$ will be granted. 

\begin{figure}[H]
	\begin{center}
		\begin{tikzpicture}

		\scriptsize
		\matrix (m)[matrix of nodes, column  sep=-15mm,row  sep=1.5mm, nodes={draw=none, anchor=center,text depth=0pt} ]{
		\textbf{\normalsize server} & & \textbf{\hspace*{20pt}\normalsize client}\\[0mm]
		& & \\
		& Initialisation of protocol \textbf{M}& \\
		& service request $R_i$& \\
		generate sub-puzzles \\
		$P[j] := g(t,x), 1\leq j \leq m$\\
		$P_i :=$ $\{\cup_{j=1}^{m}P[j],d\}$ \hfill \\
		& send puzzles & \\
		& & solve each $P[j] \in P$ \\
		& & $S_i := \{S[1 \dots m]\}$ \\
		& send solutions & \\
		verify each $S[j] \in S$ && \\ 
		& grant request $R_i$ & \\
		};
		\draw[shorten <=-1cm,shorten >=-1cm,-latex] (m-4-2.south east)--(m-4-2.south west);
		\draw[shorten <=-1cm,shorten >=-1cm,-latex] (m-8-2.south west)--(m-8-2.south east);
		\draw[shorten <=-1cm,shorten >=-1cm,-latex] (m-11-2.south east)--(m-11-2.south west);
		\draw[shorten <=-1cm,shorten >=-1cm,-latex] (m-13-2.south west)--(m-13-2.south east);
		\end{tikzpicture}
		 \vspace{10pt}
		\caption{Diagram of Proof of Work protocol}\label{tab:protocol}
	 \end{center}
\end{figure}

%[lite mer information om generering av problem. Varje problem är relativt unikt för varje request osv]


\subsection{Software}
Since one requirement on our Proof of Work system is that it minimises differences between a wide variety of devices; it needs to support both desktop and laptops with different operating systems as well as cellphones and tablets.
In order to minimise the development effort and maximise maintainability of the code base, a multi platform solution was sought for web-based simulation interface. 

A web based solution makes the application portable, but the web is not inherently stateful thus a emulation of a general service is limited. However, the advent of websockets enables a truly multi-platform proof-of-work client in html and javascript while maintaining the generality and plasticity of a natively written socket based application. 
\begin{comment}
The javascript implementation for handling the protocol is quite simple:
\jscode[firstline=57, firstnumber=57, lastline=84]{../pow.js}
The solution finding part also need to be presented:
\jscode[firstline=26, firstnumber=26, lastline=49]{../pow.js}
To trigger a request to be sent to the server we build the following function which is then registered to the onclick event of a button in the web gui:
\jscode[firstline=100, firstnumber=100, lastline=105]{../pow.js}
\end{comment}

The server side was implemented in Google's novel programming language golang\cite{golang}. The real strengths of golang in this context is actually not performance nor simplicity\footnote{But the expressiveness, clarity and performance of go programs is not to be dismissed.} but rather it's standard libraries, which in other contexts may appear immature. Golang includes standard libraries for both html template generation, http web serving as well as websockets. This package makes for an ideal platform for an application that needs to deliver the client application\footnote{HTML, CSS, Javascript and all that magic that make stuff happen in the browser} to potential clients as well as servicing clients requests in the model application reachable through the websocket interface. 

Communication is performed over a websocket per client, with a single message type which is (de-)serialised to (from) JSON. JSON is natively supported in the Javascript client, and the golang websocket library supports JSON (de-)serialisation. Thus no byte parsing of specific to our protocol had to be produced.
%\gocode[firstline=35, firstnumber=35, lastline=40]{../conn.go}
\begin{minted}[fontsize=\footnotesize]{go}
type message struct {
	Opcode, SocketId    int
	Result, Query, Hash string
	Problems            []problem.Problem
	Difficulty          problem.Difficulty
}
\end{minted}
\\
The client application is initiated by visiting the servers webpage which initiates download of the client application\footnote{The client application is built in javascript which is distributed to the client via a webpage.}. Upon requesting the application web page, the server spawns a goroutine\footnote{Goroutines are golangs built in multithreading primitive.} handling the http response writing task to deliver the application to the client.
When the client application initiates it will open a keep-alive websocket connection to the server where communication can occur. 
If the client wants the server to perform its service it will submit a request with operation code $0$, which means that the client is requesting a problem to solve in order to aquire the servers trust. The server will generate a set of puzzles tied to the specific client. To prevent an attacker from reusing solutions for multiple requests or forging problem sets the problem set $\phi_i$ is chosen by the server and stored in memory. The server will reply with the set of puzzles and an operation code $1$, telling the client to solve the problem set. The client will solve the puzzles, using a bruteforce approach since no better algorithm is known. The client will submit the solutions with an operation code $1$. Upon receiving a request with operation code 1, the server will verify the solution. If the verification succeeds the server will grant the client some cpu time, otherwise the connection will be terminated.
% \jscode[firstline=26, firstnumber=26, lastline=49]{../pow.js}
% The server will check that the solution is correct with considerable less effort:
% \gocode[firstline=142, firstnumber=141, lastline=158]{../problem/problem.go}





%\subsection{Reputation System}

