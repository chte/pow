The Denial of Service (abbr. DoS) is a class of cyber attacks directed to attack the availability of a web service. The typical DoS attack exploits the server by generating excessive memory and/or CPU utilisation of the server. Popular examples include SYN spoofing attacks which triggers the server to allocate input buffers for connections that never complete initiation as well as SSL attacks in which the server CPU is overloaded with expensive public-key decryption calculations. 
			
One of the big challenges when facing DoS attacks is to distinguish between a legitimate user and an attacker. One approach is to block senders of erroneous packages, but when malicious users pose as legitimate this method certainly fails. The objective is to ensure the availability of the service to the legitimate users without infeasible delays. 
