\subsection{DoS Attacks}
\label{text:dos}
Denial of Service attacks continues to plague internet services even if they are efficaciously protected against intrusive security breaches, as evidenced by the recent attack on Spamhaus \cite{BBC}. A denial of service attack is essentially a targeted effort to prevent a service from servicing legitimate requests by draining the underlying computer resources. Such an attack is executed by having each attacking machine performing only small load of the total work, relying on the cumulative work to overload the target system. 


%Researches have shown Proof of Work to have mitigating the effects of a DoS attacks but. To cope with a potential DoS attacks the level of difficulty of the problems is scaled proportionally with the amount stress that is put on the system. [källa på det här]

 % problem would then scale with the The solving of the problem would decrease the rate that each client would be able to issue requests to the server. This reduces the total number of requests that reach the content server in any given timeframe to a level l. l could be tuned to a level lower than the server’s inherent threshold of requests able to execute in said timeframe.

%The Denial of Service (abbr. DoS) is a class of cyber attacks directed to attack the availability of a web service. The typical DoS attack exploits the server by generating excessive memory and/or CPU utilisation of the server. Popular examples include SYN spoofing attacks which triggers the server to allocate input buffers for connections that never complete initiation as well as SSL attacks in which the server CPU is overloaded with expensive public-key decryption calculations. 
			
%One of the big challenges when facing DoS attacks is to distinguish between a legitimate user and an attacker. One approach is to block senders of erroneous packages, but when malicious users pose as legitimate this method certainly fails. The objective is to ensure the availability of the service to the legitimate users without infeasible delays. 
