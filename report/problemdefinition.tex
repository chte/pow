Proof of Work has been shown to potentially work as a prevention mechanism to at least mitigate the effects of a DoS attack\cite{subpuzzles}LÄGG IN ANNAN KÄLLA HÄR. However, scaling problem difficulty with server load may render problem difficulties that induces a significant disruption to the legitimate client. Naive proof of work implementations similair to hashcash are significantly CPU-bound\cite{hashcashbench}. Consequently, a service protected by trivial PoW schemes heavily penalises users of lesser hardware, as supported by our findings presented in table \ref{tab:flooding}, as well as \citeauthor{Tsang2008}\cite{Tsang2008}.

\begin{comment}
Proof of Work has been shown to potentially work as a prevention mechanism to at least mitigate the effects of a DoS attack without making an as assumption about the source.[källa] However, \citeauthor{LaurieC04} concluded in the paper \citetitle{LaurieC04}, that PoW on it's own, is not a feasible solution to fighting spam and denial of service attacks. This is because the classical implementation of Proof of Work does not seperate legitimate users from attackers. Hence, problems from a Proof of Work protected system would not discourage abusers of the system without having an unacceptable effect on legitimate users. 
\end{comment}