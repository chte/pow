The Internet continues to grow and has since a time back fulfilled it's early promise to enable a single source to be connected to several millions of geographically dispersed computers. However, as a consequence it has introduced security flaws of a new magnitude, allowing a single computer to potentially be attacked from millions of sources at once.

Denial of Service continues to plague internet services even if they are efficaciously protected against intrusive security breaches, as evidenced by the recent attack on Spamhaus \cite{BBC}. A denial of service attack is essentially a targeted effort to prevent a service from servicing legitimate requests by draining the underlying computer resources. Such an attack is executed by having each attacking machine performing only small load of the total work, relying on the cumulative work to overload the target system. 
\\
\\
In this paper, we focus on the computational approach to combatting denial of service attacks and to improve service survivability. This concept was originally proposed by Dwork and Naor in their report \citetitle{DworkN92} and then the idea was further researched in \citetitle{DworkGN03}. The ''proof of work'' is cryptographic in flavor and the idea is essentially to respond with a problem that is moderately hard to compute but easy to verify. Dwork and Naor originally called this a \emph{pricing function} because of it's economic origin. They introduced this concept as a way to fight e-mail spam by increasing the costs of sending spam, thus making e-mail spam economically unfeasible.

In recent research Recently, researchers have shown an increased interest in the Proof of Work (PoW) concept in order to prevent or mitigate the effects of a DoS attack on such a system. The idea is to require the clients to solve an instance of a predefined problem in order to gain access to the server’s resources. To cope with a potential DoS attacks the level of difficulty of the problems is scaled proportionally with the amount stress that is put on the system. [källa på det här]

%One of the most significant threats in server security are Denial of Service (DoS) attacks. It is essentially a targeted effort to prevent a service from functioning properly by draining the underlying computer resources. There are different levels of the attacks which either can be targeted to exploit vulnerabilities in the  TCP/IP-protocol, in the operation system that the server runs on or more specific implementations of the service. [källa på det här]
%\\
%Recently, researchers have shown an increased interest in the Proof of Work (PoW) concept in order to prevent or mitigate the effects of a DoS attack on such a system. The idea is to require the clients to solve an instance of a predefined problem in order to gain access to the server’s resources. To cope with a potential DoS attacks the level of difficulty of the problems is scaled proportionally with the amount stress that is put on the system. [källa på det här]

 % problem would then scale with the The solving of the problem would decrease the rate that each client would be able to issue requests to the server. This reduces the total number of requests that reach the content server in any given timeframe to a level l. l could be tuned to a level lower than the server’s inherent threshold of requests able to execute in said timeframe.