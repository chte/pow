This study explores a reputation based approach to scaling proof-of-work puzzle difficulties, thereby covering necessary theory of puzzle-based mechanisms specifically hash-reversal based puzzles. Furthermore, the study aims to cover explanations of the implemented protocol, simulation experiments and results of significance. 

Various types of proof-of-work protocols have recently been suggested. The study does not specifically aim to cover these, though studies of relevance are mentioned in the study. An assortment of denial-of-service attacks exists, some of which aim to disable lower level layers. This paper is limited to DoS attacks on application level layer for technical reasons. Otherwise, a proof-of-work scheme can be circumvented by an attack on a lower abstraction layer.\footnote{Application layer and abstraction layer of the OSI model.}

One important parameter of any DoS mitigation scheme is the time to live (TTL) parameter of an initiated but idle connection. The study does not cover TTL optimisation. The variable is fixed as infinite throughout this paper.
\begin{comment}


Scope: \\
The coverage of this study ..... \\
The study consists of ..... \\
The study covers the ..... \\
This study is focus on ..... \\

Delimitations: \\
The study does not cover the ..... \\
The researcher limited this research to ..... \\
This study is limited to .....


\end{comment}