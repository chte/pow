The Internet continues to grow and has since a time back fulfilled it's early promise to enable a single source to be connected to several millions of geographically dispersed computers. However, as a consequence it has introduced security flaws of a new magnitude, allowing a single computer to potentially be attacked by millions of sources at once.

A popular attack is the Denial of Service\footnote{For a brief introduction to Denial of Service attacks please see section \ref{text:dos}} (DoS) attack, which aims to restrict or disrupt the availability of the service to its users.
One attempt to counter DoS attacks and to improve service survivability is a computational approach called Proof of Work (PoW).
Proof of Work is an economic scheme that requires a service requester to perform some work in order to access a service.
The concept was originally proposed by \citeauthor{DworkN92}\cite{DworkN92} as a way to counter e-mail spam by increasing the costs of sending spam, thus making e-mail spam economically unfeasible.
\citeauthor{DworkN92} originally called this a \emph{pricing function} since it effectively puts a price on the service to the client.
PoW was reinvented by \citeauthor{Back02}, who later proposed it as a DoS preventing method in \citetitle{Back02}. 

In this paper, we examine the use of reputation based scaling of Proof of Work problem difficulties to mitigate various DoS attacks. We propose a new Reputation Based Proof of Work protection protocol (RB-PoW). The protocol was implemented via an experimental architecture and tested with a web-based simulation interface.


 
%and the idea is essentially to respond with a problem that is moderately hard to compute but easy to verify. They introduced this concept as a way to fight e-mail spam 
%In recent studies researchers have shown an increased interest in the Proof of Work (PoW) concept in order to prevent or mitigate the effects of a DoS attacks instead of fighting spam.

% The idea is to require the clients to solve an instance of a predefined problem in order to gain access to the server’s resources. To cope with a potential DoS attacks the level of difficulty of the problems is scaled proportionally with the amount stress that is put on the system. [källa på det här]


%\\
%Recently, researchers have shown an increased interest in the Proof of Work (PoW) concept in order to prevent or mitigate the effects of a DoS attack on such a system. The idea is to require the clients to solve an instance of a predefined problem in order to gain access to the server’s resources. To cope with a potential DoS attacks the level of difficulty of the problems is scaled proportionally with the amount stress that is put on the system. [källa på det här]

 % problem would then scale with the The solving of the problem would decrease the rate that each client would be able to issue requests to the server. This reduces the total number of requests that reach the content server in any given timeframe to a level l. l could be tuned to a level lower than the server’s inherent threshold of requests able to execute in said timeframe.