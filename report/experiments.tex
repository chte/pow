The reputation based PoW system's architecture and load management mechanism was tested by conducting series of simulated experiments using custom made web-based population-simulation interface. The configuration of our experiments is shown in Figure [NUMMER].

In the experiments, a fixed set of legitimate users was programmed to send requests for a service from a single server. To simulate a legitimate population each client was programmed to send the requests with stochastic based delays and thereby simulating the unpredictible rate of requests. Each request was also programmed to perform a fixed load on the server side to simulate the exectution of a service. As a result a normal work load on the server was simulated. The requests of legitimate users was mixed with a larger population of attackers to mount a DoS attack. The attackers was programmed to have lesser to zero delay inbetween requests, thus forcing the server to service more requests than a legitimate user. 

A custom made web-based monitor was used to investigate how well the RB-PoW system performed under different scenarios. The monitor uses real-time graphs to outline information about CPU utilization, requests rates and the time to solve PoW puzzles. In each run we studied the service time for attacking, legitimate and mobile users to se how the protection system affected each type of behaviour. Control parameters such as number of attackers and the settings for delays was changed between experimental runs, aswell as the parameters for server's internal reputation mechanism.

The goal of the experiments was to examine how well the reputation based PoW architecture performed in mitigating DoS attacks and at the same time serving legitimate users and users with lesser hardware. The degree of effectiveness from the experiments conducted was determined by following observations:

\begin{itemize}
\item A comparison of how many attackers that was required to launch DoS attacks with comparable levels of performance degradation, with and without reputation based PoW.

\item How much legitimate users and users with lesser hardware was affected by DoS attacks, with and without reputation based PoW protection.
\end{itemize}

\subsection{Metrics}

In order to to establish a quantitive description of our simulation experiments\footnote{See section \ref{tab:system}} and the step function of the reputation system\footnote{See section \ref{tab:reputation} and \ref{tab:behaviourmodel}}, this section will outline three sets of parameters refered as quantifiers, observables and controllables. 

\subsubsection{Attack Quantifiers}
In our Reputation Based Proof of Work system an DoS attack can be quantified by following sets of measurements: 

\emph{Service time} reflects measurement in the quality of a requested service by a client. Assuming that the service is web browsing the service time could be the time until a web page is completely rendered. If an attacker aims to DoS attack the service quality can be measured in by the average service latency for legitimate user. 

\subsubsection{Server Observables}
The difficulty of PoW problems handed out as a response to requests are based a set of selected parameters that in different ways reflects the server's resource consumption.

\emph{Number of established connections:}, keeps track of how many clients\footnote{This is actually the number of web sockets open on the server side.} in real-time that is currently connected to the server.

\emph{CPU load averages:}, tell us whether the physical CPU utilization is over or under saturated. A perfect utilization is when the CPU is busy but no process is stalled. In general load average differ from CPU usage in two significant ways:
\begin{enumerate}
\item the CPU usage measures the instantaneous snapshot while load averages measure the trend in CPU utilization.
\item the CPU usage only measures how much was active during measured timeframe while load averages take all demands for CPU into account.
\end{enumerate}
If the server has four CPUs running and the reported load average is 4.00 then the CPUs are perfectly utilized\cite{cpu}.

\emph{Request rate:} is the quantified amount of requests for a service that a single \emph{client} or a \emph{group of clients} asks for. This measurement is slightly modified to measure the average time between requests to show the amount servicing required by each client and the average of all clients in a given timeframe. By introducing a rating scheme that weights the individual requesting rate of a client with the average request rate of all clients the Reputation System can hand out problems scaled to the behaviour of each client. Thereby limiting the amount of service to a client with malicious behaviour and furthermore giving a natural limit to a DoS attack.


\subsubsection{Server Controllables}
The server resources is controlled through the Reputation System, generally it has two settings that is used to regulate.
\emph{Puzzle Difficulty:}, the difficulty of a puzzle is based on parameters in Server Observables. However, the scaling of puzzles can be tuned to respond more harshly or forgiving based on the parameters.

\emph{Connection Timeout:}, is used to control the lifetime a connection. This could be a finite duration meaning that if no solution is sent back to the server within lifetime the connection will be closed nevertheless.
However, in the scope of the simulation experiments this duration is set to infinite, see Attack Model below for explaination of our assumptions about the attacker. 
\subsection{Attack Model}
In this section, we describe the attack model of our simulation experiment. We assume a server \emph{Server}, a network with clients \emph{$\{C_i\}$}, and attackers \emph{Attacker}.
\\
\\
\noindent \textbf{Assumption 1} \indent \emph{Attacker} cannot modify any packets sent between \emph{$C_i$} and \emph{Server}.
\\
\\
This is necessary to assume, in particular that it cannot modify legitimate users packets. If Assumption 1 does not hold and packets can be modified then a denial-of-service attack can be initiated simply by corrupting all packages thus rendering any PoW protocol ineffective.
\\
\\
\\
\noindent \textbf{Assumption 2} \indent \emph{Attacker} cannot delay any packets sent between \emph{$C_i$} and \emph{Server}.
\\
\\
By similar reasons as Assumption 1 we must assume that the attacker cannot delay other clients' packets. If Assumption 2 does not hold and an attacker can delay packets arbitarily then the attacker can mount a denial-of-service attack without draining the resources of the server itself.
\\
\\
\\
\noindent \textbf{Assumption 3} \indent \emph{Attacker} cannot oversaturate the transfer layer\footnote{Transfer layer of the OSI model} or network layer\footnote{Network layer of the OSI model} of the \emph{Server}.
\\
\\
An attacker must be able to initiate a denial-of-service attack by large amount of requests to the server. However, we must also assume that the attacker cannot bring down the server by simply oversaturating the server ports or network with sheer volume of the attackers' requests.

\subsection{Faking Legitimate Users
