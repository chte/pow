The Internet continues to grow and has for quite some time enabled a single source to be connected to several millions of geographically dispersed computers. Security flaws of a new magnitude has as a consequence been introduced; enabling a single computer to be attacked by millions of sources at once. 

%\footnote{For a more indept introduction to Denial of Service attacks please see section \ref{text:dos}}
This threat also known as denial-of-service (DoS) attacks is a growing concern as these attacks have shown to disable even well-known services. Unlike other attacks the primary goal of denial-of-service attacks is to restrict or disrupt the availability of the service to its legitimate users. It is essentially a targeted effort to prevent a service from functioning properly by draining the underlying computer resources. 
%\footnote{For a more indept introduction to Proof of Work please see section \ref{text:pow}}
One attempt to counter DoS attacks and to improve service survivability is a computational approach called Proof of Work (PoW). Proof-of-work also known as \emph{Client Puzzles}\cite{dosauth, JuelsB99} is cryptographic in flavor and requires the clients to solve an instance of a predefined problem in order to gain access to the server’s resources. Proof of Work, when initially proposed, seemed like a very promising method to fighting DoS attacks but since \citeauthor{LaurieC04} published the paper \citetitle{LaurieC04} in 2004, many PoW schemes were disregarded. However, there was an aspect that \citeauthor{LaurieC04} did not consider, puzzle schemes that adapt\cite{Green,gunter}.

In this paper, we examine the use of adaptive scaling of proof-of-work problems to mitigate various DoS attacks. We propose a new Reputation Based Proof of Work protocol(RB-PoW) that scale problem difficulties based on request rate behaviour of a user.

The organisation of this paper is as follows. Section 2 explains the theory behind the reputation based proof-of-work protocol and assumptions that has to be made as well as notation of the proposed protocol. A brief view on the software testing framework is given in section 4.
An outline of important system parameters and a detailed explanation of the simulation experiments are given in Section 4. The results of the simulation experiments are presented in section 5 and a concluding discussion is provided in section 6.

% The protocol was implemented via an experimental architecture and tested with a web-based simulation interface.


 
%and the idea is essentially to respond with a problem that is moderately hard to compute but easy to verify. They introduced this concept as a way to fight e-mail spam 
%In recent studies researchers have shown an increased interest in the Proof of Work (PoW) concept in order to prevent or mitigate the effects of a DoS attacks instead of fighting spam.

% The idea is to require the clients to solve an instance of a predefined problem in order to gain access to the server’s resources. To cope with a potential DoS attacks the level of difficulty of the problems is scaled proportionally with the amount stress that is put on the system. [källa på det här]


%\\
%Recently, researchers have shown an increased interest in the Proof of Work (PoW) concept in order to prevent or mitigate the effects of a DoS attack on such a system. The idea is to require the clients to solve an instance of a predefined problem in order to gain access to the server’s resources. To cope with a potential DoS attacks the level of difficulty of the problems is scaled proportionally with the amount stress that is put on the system. [källa på det här]

 % problem would then scale with the The solving of the problem would decrease the rate that each client would be able to issue requests to the server. This reduces the total number of requests that reach the content server in any given timeframe to a level l. l could be tuned to a level lower than the server’s inherent threshold of requests able to execute in said timeframe.