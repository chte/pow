\begin{comment}
Throughout this paper attackers or users with malicious intent will be referred to as \emph{adversaries}.
\emph{Client} and \emph{user} will be used interchangeably depending on the context where the term client tend to be more software oriented in contrast to user which usually is referred from an ``in system'' perspective.
%Technically an adversary is a client in the system, however if avoidable, an adversary will not be regarded as an user because adversaries does not seek to use the system's services as intended. 
An adversary is technically a client in the system, but will not be regarded as such throughout this paper, since the intentions of the adversary is to exploit and/or interrupt the service in contrary to the legitimate client. 
%however will not be regarded as a user, since an adversary intend to exploit and/or disrupt 
\end{comment}
% \subsubsection*{Miscellaneous Abbreviations}
% A brief summary of used abbreviations follows:
A brief explanation of terminology and abbreviations used throughout this paper follows:
\begin{itemize}
\item \textbf{Client / User}: \emph{Client} and \emph{user} will be used interchangeably depending on the context where the term client tend to be more software oriented in contrast to user which usually is referred from an ``in system'' perspective.
\item \textbf{Adversary}: Attackers or users with malicious intent will be referred to as \emph{adversaries}. An adversary is technically a client in the system, but will not be regarded as such throughout this paper, since the intentions of the adversary is to exploit and/or interrupt the service in contrary to the legitimate client. 
\item {\textbf{SHA2}}: refers to the SHA256 secure hash algorithm\cite{sha2}.
\item \textbf{Puzzle}: A problem instantiation of the problem type specified in the protocol. See section \ref{text:protodesc}.
\item \textbf{Problem}: A general problem, may also refer to a Puzzle.
\item \textbf{PoW}: Abbreviation for Proof of Work.
\item \textbf{DoS}: Abbreviation for Denial of Service (attack).
%\item \textbf{}
\end{itemize}