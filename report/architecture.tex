Since one requirement of a Proof of Work system is that it minimises differences between a wide variety of devices appearing in the wild we needed to support both desktop and laptops with different operating systems as well as cellphones and tablets.
In order to minimise the development effort and maximise maintainability of the code base, a multi platform solution was sought for the client part of the demo application. 

A web based solution makes the perfect portable application, however the web is not inherently stateful and quite badly imitates an application server which keeps a constantly open socket connection with connected clients. The advent of websockets turn the whole thing around, with the help of DoS nice fellas we were able to write a truly multi-platform PoW client in html and javascript which maintains the generality and plasticity of a natively written socket based application. 

For the server side we choose to use Googles novel programming language go(\citeurl{golang}) because of its ease of use as well as performance. The real strength of golang in this context is actually it's standard libraries, which in other contextes may appear immature. Golang actually has standard libraries for both http template generation, web serving as well as websockets. This package makes for an ideal platform for an application that needs to deliver the client application\footnote{HTML, CSS, Javascript and all that magic that make stuff happen in the browser} to potential clients as well as servicing clients requests in the model application reachable through the websocket interface. 

To further ease our programming task, we choose to communicate over the websocket with a single message type which is (de-)serialised (from) to JSON. JSON is natively supported in the Javascript client, and the golang websocket library supports JSON (de-)serialisation which makes for a nice pairing where we have to write absolutely no byte parsing for our protocol.
\gocode[firstline=35, firstnumber=35, lastline=40]{../conn.go}



