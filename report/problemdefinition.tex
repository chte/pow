Proof of Work has been shown to potentially work as a prevention mechanism to at least mitigate the effects of a DoS attack\cite{subpuzzles}LÄGG IN ANNAN KÄLLA HÄR. However, scaling the problem difficulty with server load has proven to render problems so difficult that the problems induce a significant disruption to the legitimate client [VÅRA RESULTAT ELLER NGN ANNANS, KÄLLA PLIX]. Other problem with naive proof of work implementations such as hashcash is that it is largely CPU-bound\cite{hashcashbench}. This 


\begin{comment}
Proof of Work has been shown to potentially work as a prevention mechanism to at least mitigate the effects of a DoS attack without making an as assumption about the source.[källa] However, \citeauthor{LaurieC04} concluded in the paper \citetitle{LaurieC04}, that PoW on it's own, is not a feasible solution to fighting spam and denial of service attacks. This is because the classical implementation of Proof of Work does not seperate legitimate users from attackers. Hence, problems from a Proof of Work protected system would not discourage abusers of the system without having an unacceptable effect on legitimate users. 
\end{comment}