\subsection{Feasibility}
\subsection{Statistical Confidence}
An important question of our study is, \emph{when do we trust our data?} The results presented in Section \ref{result} was collected from the simulation experiments run through the web-based simulation interface. The data was collected as samples during a certain time-frame during the experiments, dividing data into the categories of \emph{adversaries}, \emph{legitimate users} and \emph{mobile device users}. However,  can we be certain that the calculated averages of our samples represent the average of the whole population?

The answer lies in statistical mathematics. To bring confidence in the presented results it is important that a average of the sample data can, with a certain propability, be found within a interval of confidence, also known as confidence interval. Hence, knowing this interval enables the possiblity to infer that the average of one sample is significantly different from another, as long as the confidence interval of the averages do not overlap.

One fundamental requirement of finding these confidence intervals is that the distrubution of the samples is known. However, the distribution of our the result data in our simulation experiments is likely a sum of pascal and unknown distributed variables. Because of the uncertainty a bit of magic is required to solve this problem. 

\subsubsection{Arithmetic Averages Have a Bit of Magic}
That bit of magic is the \emph{Central Limit Theorem}. The theorem states that, given a sufficiently large sample of indentically distributed independent variables,\footnote{The central limit theorem generally takes effect when samples is larger than 30.}, each with a well-defined mean and well-defined variance\footnote{This implies that both the mean the variance should be finite.}, will be approximately normally distributed\cite{gunnar}.

From this theorem two implications can be drawn that is relevant for our test data:
\begin{itemize}
\item A random sample scan be taken from any population, in our case samples of the simulation experiments, even if the simulation data is not normally distributed and assume it to be approximately normally distributed.

\item The theorem also allows assumptions to be made about the sample data of the simulation regardless of the entire simulation data. Thereby, an interval estimation can be made about the true average of the simulation data with only sample data.
\end{itemize}

 \subsubsection{Confidence in Test Results}
The probability that the average of whole population falls within the interval of the result data is either 1 or 0 - the interval captures the average or it doesn't\cite{gunnar}. However, with a 99\% confidence it is safe to assume that the average of the whole population actually falls within the confidence interval of the result data. 


\subsection{Lessons learned}
\subsection{Suggested Directions for Future Research}